\documentclass{article}

% Language setting
% Replace `english' with e.g. `spanish' to change the document language
\usepackage[spanish]{babel}

% Set page size and margins
% Replace `letterpaper' with `a4paper' for UK/EU standard size
\usepackage[letterpaper,top=2cm,bottom=2cm,left=3cm,right=3cm,marginparwidth=1.75cm]{geometry}

% Useful packages
\usepackage{amsmath}
\usepackage{graphicx}
\usepackage{amssymb}
\usepackage[colorlinks=true, allcolors=blue]{hyperref}

\title {\textbf{El algoritmo de dos fases para el reconocimiento de actividades en el contexto de la Industria 4.0 y los procesos dirigidos por humanos
}}
\author{Borja Bordel1, Ramón Alcarria1, Diego Sánchez-de-Rivera1 1Universidad Politécnica de Madrid, Madrid, España bbordel@dit.upm.es, ramon.alcarria@upm.es, diegosanchez@dit.upm.es}

\begin{document}
\maketitle

\begin{abstract}
Los futuros sistemas de industria, la revolución conocida como la Industria 4.0, están considerando integrar gente en el mundo de Internet como potenciales prosumidores (entre proveedores de servicios y consumidores). En este contexto, el trabajo dirigido por humanos aprece como una realidad esencial y los instrumentos para generar bucles  de información que se retroalimentan entre el subsistema social (personas) y el subsistema cibernético (componentes tecnológicas) son necesarios. A pesar de que muchos instrumentos distintos han sido propuestos, a día de hoy las técnicas de reconocimiento de patrones son las más prometedoras. Sin embargo, estas soluciones presentan algunos problemas pendientes. Por ejemplo, estos dependen del hadware seleccionado para obtener información de los usuarios; o presentan límites en el proceso de identificación. Para abordar esta situación, en este paper se propone un algoritmo de dos fases para integrar a las personas en los sistemas de la Industria 4.0. El algoritmo define acciones complejas como una composición de simples movimientos. Las acciones complejas son reconocidas utilizando los Movimientos Ocultos de Markov, y solo los movimientos simples se reconocen a través la DWT . De este modo, solo los movimientos dependen de los recursos de hadrdware empleados para capturar información, y se incrementa  la precisión en el reconocimiento de acciones complejas. Una validación experimental será también llevada a cabo para evaluar y comparar la interpretación de los resultados propuestos.
Palabras clave: Industria 4.0; reconocimiento de patrones; DWT; Inteligencia Artificial; Modelos Ocultos de Markov.
\end{abstract}

\section{Introducción}
La Industria 4.0 [1] se basa en el uso sistemas Ciber-Físicos (uniones de procesos físicos y cibernéticos) [2] como componente tecnológica principal para futuras soluciones digitales, sobre todo  ( pero no únicamente) en el ámbito industrial. Por lo general, la digitalización ha causado, al final, el remplazo de los métodos tradicionales de trabajo por nuevos instrumentos digitales.
Por ejemplo, los trabajadores en producción en cadena fueron sustituidos por robots durante la tercera revolución industrial.
Sin embargo, algunas aplicaciones industriales no pueden basarse en soluciones tecnológicas, por lo que la mano de obra se vuelve esencial [3].  Los productos hechos a mano son un ejemplo que reflejan la importancia del trabajo humano. Estos sectores de la industria, en cualquier caso, deben integrarse en la cuarta revolución industrial. Desde la unión de los Sistemas Ciber-Físicos (CPS) y el papel de las personas como proveedores de servicios (trabajos activos), surgen los CPS humanizados[4]. En estos nuevos sistemas, los procesos dirigidos por humanos están permitidos; los procesos[5] i.e. que son conocidos, ejecutados y controlados por gente (aunque serán supervisados a través de mecanismos digitales).
Para una verdadera integración entre la gente y la tecnología, que elemine los procesos de ejecución humana o tecnológica, para lo que se requieren técnicas de extración de información. Durante los últimos años se han citado distintos métodos posibles, pero a día de hoy las técnicas de reconocimiento de patrones parecen las más promentedoras.
El uso de la IA, modelos estadísticos y recursos similares ha permitido un increíble desarrollo de los métodos de reconocimiento de patrones, aunque todavía quedan retos pendientes.
Inicialmente, las técnicas de reconocimiento de patrones dependen del hardware subyacente para la captura de información. La estructura y la memorización varía si (por ejemplo)   empleamos sensores de infrarrojos en vez de acelerómetros. Es muy problemática la velocicidad de evolución de las tecnologías hardware con respecto al software. 
Por otro lado, existe un límite en la precisión del reconocimiento de procesos. De hecho, con las acciones humanas se vuelve más complicado, ya que se requieren más variables y modelos complejos para su reconocimiento. Esta aproximación genera grandes problemas de optimización con un error que aumenta al incrementar el número de variables, lo que provoca un decrecimiento del índice de reconocimiento[6]. En conclusión, las matemáticas (-------) aportan una cierta precisión al proceso de reconocimiento de patrones dando medidas para estudiar. Para evitar esta situación, deben ser consideradas menos variables. lo que reduce la complejidad del problema a la hora de analizarlo, una solución que no es aceptable en los ámbitos de la industria donde se desarrollan actividades de producción complejas.
Por lo tanto, el objetivo de este paper es describir un nuevo algoriitmo de reconocimiento de patrones dirigido a estos dos problrmas. El mecanismo propuesto define las acciones como una composición de movimientos simples. Estos se reconocen utilizando las técnicas de Deformación Dinámica del Tiempo (DTW) [7]. Este proceso depende del hardware empleado para la captura de información; pero las DTW son flexibles y la actualización del repositorio de reconocimiento es suficiente para configurar el algoritmo completo. Por otro lado, las acciones complejas son reconocidas como una combinación de movimientos simples a través de los Modelos Ocultos de Markov. HMM) [8]. Estos son totalmente independientes de la tecnología hardware , ya que solo consideran acciones simples. Estas dos fases  se aproximan y reducen la complejidad de los modelos, incrementando la precisión y el éxito del índice ene en el reconocimiento de patrones.
El resto del paper se organiza como sigue: en la Sección 2 se describe el funcionamiento del reconocimiento de patrones para actividades humanas; en la Sección 3 se describe la solución propuesta, incluyrndo las dos fases definidas; la Sección 4 presenta una prueba  experimental utilizando un escenario real; y la en la sección se concluye el paper.
\section {Validación experimental: implementación y resultados}
\section {Algoritmo de reconocimiento de patrones en dos fases}
\subsection{Reconocimiento de movimientos simples: Dinámicas de Deformación del Tiempo}
\subsection{Reconocimiento de movimientos complejos: Modelos Ocultos de Markov}
\section {Validación experimental: implementación y resultados}
Con el fin de evaluar el cumplimiento de la solución propuesta, se ha designado una validación experimental y se ha llevado a cabo. Un escenario industrial ha sido imitado en algunas salas grandes de la Universidad Politécnica de Madrid. El escenario representó a una compañia tradicional de producción de productos hechos a mano. En particular, imitaron un pequeño PCB(placa de circuito impreso) productor.

Para conseguir información sobre el comportamiento de las personas, a varios participantes se les proporcionó guantes cibernéticos, incluyendo acelerómetros y un lector RFID. Los objetos cercanos al escenario fueron identificados con una etiqueta RFID, por lo que la plataforma de hardware propuesta puede identificar la posición de la mano y los objetos con los que las personas han interactuado. 

Una lista de 12 actividades diferentes fue diseñada y definida usando la tecnología propuesta. La tabla 1 describe las 12 actividades definidas, además de una breve descripción sobre ellas.



18 personas estuvieron involucradas en el experimento. Las personas fueron solicitadas para realizar las actividades con números aleatorios. El orden verdadero, así como el de las actividades es reconocido y almacenado por un proceso de software supervisor. La tasa global de éxito para todas las soluciones ha sido evaluada, identificando, el mismo éxito para cada una de las distintas fases.

Para evaluar las mejoras obtenidas en comparación con las soluciones similares ya existentes, la misma secuencia de datos físicos ha sido empleada para alimentar la solución de un patrón estándar de reconocimiento basado solo en HMM. Usando un software de procesamiento de datos específicos, algunos resultados relevantes han sido obtenidos.

La figura 4 representa la tasa media de éxito para 3 casos: la solución global, la primera fase(DTW) y la segunda fase (HMM). Además, la tasa de éxito para el tradicional HMM-based acercamiento está incluido. Como se puede ver, la tecnología sugerida es, globalmente, alrededor de un 9 \% mejor que  las técnicas tradicionales de reconocimiento de patrón basadas en HMM exclusivamente. Además, la primera fase(basada en DTW) es alrededor de un 20 \% peor que la segunda fase(HMM) que es significativo que las DTW técnicas son más débiles por defecto.


\section {Conclusiones y futuros trabajos}
En este paper presentamos un nuevo patrón algorítmico de reconocimiento para integrar individuos en los sistemas de la Industria 4.0 y los procesos dirigidos por humano. El algoritmo define actividades complejas como una composición de movimientos simples. Las actividades complejas son reconocidas empleando los Modelos Ocultos de Maekov, y los movimientos simples son reconocidos utilizando las DWT. Con el fin de activar el implemento de este algoritmo incrustado en objetos pequeños, se seleccionan configuraciones ligeras. Una prueba experimental de esto es llevarlo a cabo, y los resultados muestran una mejora global en un exitoso índice del 9%.
Los futuros trabajos considerarán metodologías más complejas para el procesamiento de datos, y la comparación para configuraciones distintas de la propuesta serán evaluadas.
Además, la propuesta será analizada en diferentes escenarios. 

\end{document}